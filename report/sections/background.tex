Racket is a programming language \cite{racket}. Included with every download of Racket is an integrated development environment (IDE), DrRacket. DrRacket is the ``official'' IDE for Racket and is maintained by the same core group of contributors. 

\newfigure{drracket}{DrRacket}{width=10.5cm}{h!}

DrRacket's user interface is divided into two main areas: the definitions window, by default on top, and the interactions window. The definitions window contains the source file while the interactions window displays the program's output. Above the definitions window are a horizontal list of buttons. Included in this list is the ``run'' button, which, when pressed, executes the program in the definitions window. See figure \ref{fig:drracket} for a sample DrRacket environment.

DrRacket's default code coverage can be enabled by selecting ``Syntactic Test Suite Coverage'' in the ``Choose Language'' dialog. Once code coverage has been enabled, it is collected every time the program is executed by clicking the ``run'' button. The code coverage information is then displayed by coloring covered code green and uncovered code red. (Figure~\ref{fig:drracket}). However, if the entire program is covered, in which case no code coloring is done.

In order to determine which sections of code to color, DrRacket must keep track of which expressions of have been executed. DrRacket does this by adding additional code, hidden from the user, before the program is run. This added code surrounds every expression with an integer variable that represents the number of times the expression has been executed. So, if the variable is 0 the expression has not been executed. Then, by examining these variables, it can be determined which sections of code have not been executed. Additionally, these variables can be used to profile which sections of code are heavily used, but is not the focus of this senior project and will not be covered. Figure \ref{fig:default-flow} shows the simplified program flow for code coverage and code highlighting in DrRacket before the solution implemented by this senior project.

\newfigure{default-flow}{DrRacket Code Coverage Flow}{width=7cm}{h!}

DrRacket can be extended in two ways, either through a new collection or a PLaneT package. Collections are tightly coupled with DrRacket and are often distributed with the default installation. PLaneT packages, on the other hand, are more modular. They can be downloaded and automatically installed form a central package repository. Both methods can be used to create new tools for DrRacket. At start up DrRacket looks for tools by reading \emph{info.rkt} files found in collections and installed PLanT packages \cite{plugin}. These tools can then extend DrRacket through the use of mixins. Every major component of DrRacket is a mixin, from the ``Run'' button to tabs. 
