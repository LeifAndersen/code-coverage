There are many software development tools and practices that help programmers produce more reliable code, faster and efficiently. One of these tools is code coverage. Code coverage describes the amount of the program that has been tested \cite{lasse}. It is useful in helping programmers write tests that execute all, or most, of the source code. Code coverage can be seen as an indirect measure of test quality \cite{lasse}.

DrRacket, the Racket integrated development environment (IDE), includes limited code coverage. It only displays code coverage highlighting on a single file. That is, if a program's test cases are in a separate file, only that file will have coverage highlighting. This is not helpful when trying to determine the code coverage of the entire program, since only the test case file has code coverage. This creates the main goal of this senior project, extending DrRacket's code coverage to multiple files. 

DrRacket's ode coverage highlighting only gives a general feeling of the amount of covered versus uncovered code. No details about the code coverage, such as the percent of a file that is covered, is provided to the user. This is the first additional goal, provide the user with concrete code coverage information.

The final additional goal was to implement the extension in such a way that minimal modifications to DrRacket's source code were required. By limiting the modifications to DrRacket's source code distribution would be easier and faster. Additionally, limited modification would reduce the amount of new bugs introduced to DrRacket.

The Multi-file Code Coverage Tool was created to fulfill three goals: extend DrRacket's code coverage highlighting to multiple files, provide concrete code coverage information, and limit the amount of modifications to DrRacket's source code.