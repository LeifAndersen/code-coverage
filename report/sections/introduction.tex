There are many software development tools and practices that help programmers produce more reliable code, faster and efficiently. One of these tools is code coverage. Code coverage describes the amount of the program that has been tested \cite{lasse}. It is useful in helping programmers write tests that execute all, or most, of the source code. Code coverage can be seen as an indirect measure of test quality \cite{lasse}.

DrRacket, the Racket integrated development environment (IDE), includes limited code coverage. It only displays code coverage highlighting on a single file. That is, if a program's test cases are in a separate file, only that file will have coverage highlighting. This is not helpful when trying to determine the code coverage of the entire program, since only the test case file has code coverage. Additionally, code coverage highlighting only gives a general feeling of the amount of covered versus uncovered code.

The main focus of this senior project was to extend DrRacket's code coverage highlighting to multiple files. Two additional goals where: one, to provide some concrete information on the amount of covered code in a program; and two, write the code in such a way that integrating it with DrRacket required either minimal or no modifications to DrRacket's source code.