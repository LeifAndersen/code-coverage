Before attempting to implement a solution to extend DrRacket's code coverage to multiple files, research was required to better understand how DrRacket implements and uses code coverage. This research consisted primarily of modifying small sections of DrRacket and then examining the results. While this process was not particularly fast, it did eventually reveal the information needed to successfully implement the tool.

During the research, the following facts where discovered: one, the organization of DrRacket's user interface follows the hierarchy seen in figure \ref{fig:gui-classes}; two, DrRacket generates a test coverage info hash table that contains information for all uncompiled required files; and three, the test info hash table persists after code coloring has been applied.

DrRacket's user interface is composed of four main classes, seen as boxes in figure \ref{gui-classes}. The highest class is the \emph{frame}. The \emph{frame} is an entire DrRacket environment as seen in figure \ref{fig:drracket}. Each frame has at least one \emph{tab}. A \emph{tab} always contains a definitions window, where the source code is visible and editable. A \emph{tab} may also contain an interactions window. Additionally, the frame group contains a list of all currently open \emph{frames}.

\newfigure{gui-classes}{DrRacket User Interface Classes}{width=6cm}{h!} 

%add detail about testing methods
DrRacket generates a hash table, called \emph{test-coverage-info} internally, when the program is run. Included in the hash table are expressions, with their respective source code location, and whether or not they have been evaluated. By examining the contents of the test coverage info hash table, for a program that includes multiple files, it was discovered that coverage information is collected for all uncompiled files. So while DrRacket's default code coverage only applies code coloring to the active file, the test coverage info hash table has the information needed to color additional files. Code coverage can not be collected for compiled files because their code can not be expanded to track executed expressions, see figure \ref{fig:default-flow}.

The test coverage info hash table can be sent to a method of DrRacket's \emph{tab} class, called \emph{show-test-coverage-annotations}, which does code coloring for that tab based on the hash table. This hash remains available for use after code coloring has been completed. In order to clear the test coverage info hash table an additional method, \emph{clear-test-coverage}, must be called. 

